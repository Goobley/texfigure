\documentclass[]{article}

\usepackage{pythontex}
% % % % % % % % % % % % % % % % % % % % % % % % % % % % % % % % % %
% PythonTeX Bug Fix % % % % % % % % % % % % % % % % % % % % % % % %
% % % % % % % % % % % % % % % % % % % % % % % % % % % % % % % % % % 
% pytexbug fix for context in customcode.
\makeatletter
\renewenvironment{pythontexcustomcode}[2][begin]{%
	\VerbatimEnvironment
	\Depythontex{env:pythontexcustomcode:om:n}%
	\ifstrequal{#1}{begin}{}{%
		\ifstrequal{#1}{end}{}{\PackageError{\pytx@packagename}%
			{Invalid optional argument for pythontexcustomcode}{}
		}%
	}%
	\xdef\pytx@type{CC:#2:#1}%
	\edef\pytx@cmd{code}%
	% PATCH \def\pytx@context{}%
	\pytx@SetContext
	% END PATCH
	\def\pytx@group{none}%
	\pytx@BeginCodeEnv[none]}%
{\end{VerbatimOut}%
\setcounter{FancyVerbLine}{\value{pytx@FancyVerbLineTemp}}%
\stepcounter{\pytx@counter}%
}%
\makeatother
% % % % % % % % % % % % % % % % % % % % % % % % % % % % % % % % % %

\usepackage{graphicx}
\usepackage{pgf}
\usepackage{float}
\usepackage{import}
\usepackage[margin=3cm,noheadfoot]{geometry}

%opening
\title{An Example of Astropy and PythonTeX Integration}
\author{Stuart Mumford}

\setpythontexcontext{figurewidth=\the\columnwidth}
\newcommand{\includepgf}[1]{\IfFileExists{#1}{\input{#1}}{}}

\begin{document}
\begin{pythontexcustomcode}{py}
import texfigure
texfigure.configure_latex_plots(pytex)
pytex.formatter = texfigure.repr_latex_formatter
\end{pythontexcustomcode}
\section{Figure Manager}

The figure manager is the primary component of PythonTeX, it makes it easier to
generate and work with figures generated in PythonTeX code blocks from within a
LaTeX document.

\begin{pycode}[chapter2]
from texfigure import Manager
ch2 = Manager(pytex, './')
\end{pycode}


\begin{pycode}[chapter2]
import matplotlib.pyplot as plt
import numpy as np

fig = plt.figure(figsize=(4,4))
ax = fig.gca()
ax.imshow(np.random.randn(20,20))

lfig = ch2.save_figure('rand', fig, fext='.pgf')
lfig.caption = 'Some randomness'
\end{pycode}

\py[chapter2]|ch2.get_figure('rand')|

You can now reference the figure above, \ref{fig:rand}.

\end{document}
